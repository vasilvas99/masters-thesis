\section{Заключение}
В настоящата дипломна работа е предприет модерен прочит на една класическа област от изследвания - кристалния растеж и нестабилностите, илюстриран с помощта на числени методи и оригинални резултати. Освен общите положения, известни в литературата са разгледани два подхода за моделиране - окрупнен (нормалния растеж на кристалната стена) и микроскопски (тангенциален, в контекста на групиране на повърхностни стъпала или т.нар. \textit{step bunching}).

В рамките на макроскопския подход е получен нов модел за степента на превръщане - $\aDg$, за който беше построен протокол за намиране на параметрите му ($D, g$) за конкретни експериментални данни. Беше направено кръстосано валидиране с широко използвания модел на Johnson-Mehl-Avarami-Kolmogorov, моделът на Ричардс и експериментални данни от литературата.

При микроскопския подход е дадено по-подробно описание на групиране на стъпалата на кристалната повърхност при растеж или изпарение (step bunching). Даден е адвективно-дифузионно-реакционен модел за динамиката на адатомите в разтвора, без допълнителното му изследване. Основният фокус тук е върху опростени ОДУ модели - \textbf{MM}, \textbf{LW}, \textbf{TE} - тяхното подходящо обезразмеряване и изследването на времевия скейлинг на броя стъпала в групата $N \propto t^\beta$. 

Накрая е изследван трети модел - на Pimpinelli-Tonchev-Videcoq-Vladimirova, който е едномерно ЧДУ за еволюцията на височината на повърхността с времето.  PTVV е изследван за инвариантност под разтягане и от това е получен общ израз за морфологичния параметър $\beta$, както и връзка между $h$, автомоделната функция $f(s)$ и броя стъпала в групата N. Обезразмерен и изследван за инвариантност под разтягане е и по-простият частен случай на PTVV - уравнението на Стоянов и Тончев, при което началният размерен вид значително е опростен, както и някои от по-сложните резултати на оригиналният труд на Стоянов и Тончев са получени директно от този анализ без да се налага интегриране на уравнението. Получен е времевият скейлинг ($\beta$) на броя стъпала в групата, което пък сочи конкретно на кой избор на параметрите $k, \rho$ на PTVV е еквивалентно ST.
